\clearpage
\chapter*{ABSTRAK}
\addcontentsline{toc}{chapter}{ABSTRAK}

%taruh abstrak bahasa indonesia di sini
\begin{center}
  \center
  \begin{singlespace}
    \bfseries \MakeUppercase{\thetitle}

    \normalfont\normalsize
    Oleh:
    \bfseries \theauthor
  \end{singlespace}
\end{center}

% \vspace{1cm}

\begin{singlespace}
  Perkembangan dunia kriptografi saat ini cukup cepat. Salah satu contoh dari perkembangan kriptografi saat ini adalah \emph{cipher} dengan kunci dinamis. Terdapat beberapa penelitian yang membahas terkait hal ini, hanya saja saat ini belum ada protokol yang menerapkan \emph{cipher} ini secara nyata. Tugas akhir ini memiliki tujuan untuk mengimplementasikan algoritma enkripsi AES dengan kunci blok dinamis berbasis \emph{chaos} pada protokol TLS. Tugas akhir ini juga bertujuan untuk memilih sistem \emph{chaos} yang tepat untuk diimplementasikan pada protokol TLS. Tugas akhir ini juga menguji keamanan dari \emph{cipher} dan protokol yang dibangun. Sistem \emph{chaos} yang digunakan sebagai pembangkit kunci pada penelitian ini adalah \emph{Sine-Henon Map}. Hal ini dikarenakan sistem \emph{chaos} ini memiliki kualitas keteracakan yang baik dan menawarkan \emph{forward unpredictability}. Mode blok yang digunakan pada penelitian ini adalah \emph{counter} berbasis sistem \emph{chaos}. Hal ini diharapkan dapat meningkatkan kualitas dari hasil enkripsi. Implementasi protokol ini dilakukan pada protokol TLSv1.2. Proses implementasi dilakukan dengan menambahkan \emph{cipher suite} pada protokol TLSv1.2. Implementasi dilakukan dengan membuat sebuah pustaka. Pustaka ini dibangun dengan bahasa pemrograman Python. Pengujian  \emph{cipher} yang dilakukan adalah uji keacakan, uji ketahanan, dan uji skenario. Pengujian protokol dilakukan dengan uji skenario. Uji keacakan dilakukan dengan memanfaatkan NIST Statistical test. Uji ketahanan dilakukan dengan pengujian CCA dan MAD. Uji skenario dilakukan berdasarkan kondisi normal dan kondisi dalam serangan. Hasil pengujian menunjukan nilai acak yang dihasilkan oleh sistem \emph{chaos Sine-Henon map} memiliki kualitas yang setara dengan CSPRNG \texttt{urandom} pada sistem Linux. Hasil pengujian melalui analisis keteracakan menunjukan bahwa \emph{cipher} AES dengan kunci blok dinamis memiliki kekuatan yang setara dengan AES dengan kunci statis pada mode blok \emph{counter}. Hasil pengujian ketahanan yang dilakukan menunjukan bahwa AES dengan kunci blok dinamis memiliki kekuatan yang setara dengan AES dengan kunci statis. Hasil pengujian skenario  menunjukan bahwa protokol dan \emph{cipher} dapat bekerja dengan baik memanfaatkan \emph{cipher} kunci blok dinamis. Selain itu, protokol ini juga mampu bertahan khususnya terhadap serangan \emph{replay}, serangan MITM, dan serangan \emph{tampering}.
  
  \textbf{\textit{Kata kunci: Sine-Henon Map, TLS, AES, Sistem Chaos, Cipher Kunci Dinamis}}
\end{singlespace}
\clearpage
