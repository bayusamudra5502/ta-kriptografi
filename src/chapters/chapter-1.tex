\chapter{Pendahuluan}

\section{Latar Belakang}
Masalah keamanan informasi merupakan permasalahan yang telah muncul semenjak dahulu kala. Hal ini dikarenakan orang-orang ingin memastikan informasi yang dikirimkan hanyalah boleh diakses oleh orang yang memiliki wewenang untuk membukanya. Permasalahan keamanan informasi saat ini tentu saja masih menjadi yang sangat relevan. Berdasarkan data yang diunggah pada \textcite{pusiknaspolri_cybercrime_2022}, penindakan kasus kejahatan siber pada tahun tersebut meningkat sangat drastis bila dibandingkan dengan tahun sebelumnya. Hal ini memberikan tanda bahwa kejahatan siber terus berkembang. Hal ini juga diperparah dengan perkembangan teknologi dan teknik untuk melakukan peretasan semakin canggih. Oleh karena itu, diperlukan pengembangan teknik untuk melindungi data agar semakin aman.

Untuk mengamankan sebuah informasi, dapat dilakukan berbagai teknik. Salah satu teknik yang sangat penting untuk mengamankan informasi adalah kriptografi. Ilmu terkait kriptografi sudah mulai dikembangkan sejak masa dahulu kala. Hal ini dapat diamati dari kemunculan adanya \emph{Caesar Cipher}. Kemunculan dari ilmu kriptografi disebabkan oleh adanya keperluan untuk menjaga data tetap rahasia dan hanya boleh diakses oleh orang yang memiliki wewenang. Semakin berkembangnya teknologi informasi, ilmu kriptografi ikut berkembang. Pada masa kini, telah terdapat beberapa  algoritma kriptografi yang dapat digunakan untuk pengiriman pesan. Menurut \textcite{munir2019}, beberapa algoritma tersebut dapat diklasifikasikan berdasarkan jumlah kunci yang terlibat dalam melakukan enkripsi dan dekripsi, yaitu kriptografi kunci simetri dan kriptografi kunci asimetris.

Menurut \textcite{halak2022}, penggunaan algoritma kunci asimetris lebih berat dibandingkan dengan penggunaan algoritma kunci simetris. Hal ini menyebabkan banyak protokol yang memanfaatkan kunci simetris dalam pertukaran pesan. Penggunaan kunci asimetris hanya digunakan untuk proses otentikasi pada saat awal dari komunikasi. Contoh dari protokol yang memanfaatkan pola seperti itu adalah \emph{Transport Layer Security} (TLS).

Alternatif dari penggunaan algoritma kunci asimetris adalah penggunaan kunci simetris. Penggunaan kunci simetris memiliki kelebihan dalam hal kecepatan proses enkripsi dan dekripsi. Saat ini, perkembangan teknologi kunci simetris juga semakin berkembang. Salah satu teknik yang dapat digunakan untuk meningkatkan keamanan dari kunci simetris adalah dengan memanfaatkan kunci simetris dinamis. Kunci simetris dinamis adalah kunci simetris yang berubah setiap kali proses enkripsi dan dekripsi dilakukan. Hal ini tentu saja akan meningkatkan keamanan dari pesan yang dikirimkan karena \emph{cipher} seperti ini menghilangkan pola statistik dengan mengganti kunci enkripsinya. Menurut \textcite{munir2019}, \emph{cipher} seperti ini menerapkan prinsip \emph{confussion} dari Shannon.

Terdapat beberapa penelitian yang merancang algoritma enkripsi dinamis untuk mengenkripsi data yang dikirimkan. Pada penelitian yang dilakukan oleh \textcite{singh2019}, kedinamisan terletak pada arsitektur internal AES dalam satu \emph{round}. Penelitian ini mencoba untuk membentuk kedinamisan hasil enkripsi dengan melakukan permutasi pada S-Box, menggunakan \emph{irreducible polynomial} lain yang tersedia, serta mengganti nilai konstanta \emph{affine}. Dengan mengganti internal dalam sebuah \emph{round}, proses enkripsi tentu akan menjadi semakin kompleks. Selain itu, proses enkripsi tidak dapat dioptimasi terutama apabila CPU mendukung instruksi set yang mendukung proses menjalankan round menggunakan AES dikarenakan internal round telah berbeda.

Penelitian yang dilakukan oleh \textcite{lin2021} memanfaatkan teori \emph{chaos} untuk membangkitkan kunci yang dinamis. Selain itu, penelitian ini juga membahas terkait sinkronisasi sistem \emph{chaos} memanfaatkan \emph{sliding mode control}. Proses sinkronisasi memanfaatkan teknik tersebut memiliki sebuah kekurangan bila diimplementasikan untuk sistem enkripsi ujung ke ujung, yaitu jumlah pengiriman pesan yang perlu dilakukan selama sinkronisasi tidak konstan. Selain itu, penelitian yang dilakukan oleh \textcite{lin2021} tidak menjelaskan terkait protokol pengiriman pesan.

Dari beberapa permasalahan dan studi yang telah disebutkan, penelitian ini akan membahas implementasi algoritma enkripsi simetris dinamis dalam pengiriman pesan memanfaatkan sistem \emph{chaos} pada protokol TLS. Penelitian ini akan membahas terkait mekanisme sinkronisasi \emph{state} melalui fitur yang ada pada protokol TLS. Penelitian ini juga akan membahas terkait pengujian keamanan dari protokol yang diusulkan.

\section{Rumusan Masalah}
Berdasarkan latar belakang yang telah dijelaskan sebelumnya, dapat disimpulkan bahwa terdapat permasalahan terkait pengiriman memanfaatkan algoritma enkripsi kunci simetris dinamis pada protokol TLS. Untuk menyelesaikan permasalahan tersebut, dibentuk beberapa rumusan masalah:
\begin{enumerate}
  \item Bagaimana cara melakukan enkripsi pesan memanfaatkan Algoritma AES kunci blok dinamis yang aman?
  \item Bagaimana cara mengimplementasikan Algoritma AES kunci blok dinamis pada protokol TLS dengan aman?
\end{enumerate}

\section{Tujuan}
Penelitian ini dilakukan dengan tujuan sebagai berikut:

\begin{enumerate}
  \item Membangun sebuah protokol komunikasi berbasis TLS memanfaatkan algoritma kunci blok simetris.
  \item Menguji pengiriman pesan pada melalui protokol TLS memanfaatkan algoritma kunci blok simetris.
  \item Menguji keamanan \emph{cipher} yang telah dibuat dan implementasi dari protokol TLS.
\end{enumerate}

\section{Batasan Masalah}
Dalam penelitian ini, terdapat batasan masalah yang digunakan, yaitu sebagai berikut:
\begin{enumerate}
  \item Komunikasi dilakukan hanya oleh dua entitas.
  \item Protokol TLS berjalan di atas jaringan yang reliable.
  \item Proses implementasi protokol TLS hanya dilakukan hingga tahap pembuatan prototype.
\end{enumerate}

\section{Metodologi}
Metodologi yang digunakan pada penelitian ini terdapat empat tahap. Berikut ini merupakan metodologi yang digunakan pada penelitian ini:
\begin{enumerate}
  \item Analisis Masalah\\
  Penelitian pada tahap ini berfokus pada permasalahan protokol yang telah dibentuk saat ini. Pada tahap ini juga akan dijelaskan permasalahan yang mungkin muncul apabila mengimplementasikan sistem enkripsi simetris dengan kunci blok dinamis pada protokol TLS. Pada tahap ini juga akan dijelaskan terkait kebutuhan dari protokol TLS yang akan dibangun.

  \item Perancangan protokol komunikasi\\
  Setelah melakukan analisis permasalahan, perlu dilakukan perancangan solusi berupa skema protokol TLS dan sistem enkripsinya. Pada bagian ini, Penelitian mengkaji jenis mode blok yang digunakan pada protokol yang diusulkan. Pada tahap ini juga, penelitian mengkaji terkait pembangkitan \emph{chaos state} berdasarkan dari kunci sesi yang telah dibentuk pada protokol TLS.

  \item Pembangunan prototype protokol komunikasi\\
  Pada tahap ini, prototype protokol komunikasi yang telah dirancang sebelumnya dibangun. Pembangunan prototype dilakukan berdasarkan perancangan desain yang telah disusun pada bagian sebelumnya. Prototype dibangun dalam bentuk program kecil yang dapat mengirimkan pesan secara bolak balik sebagai bukti bahwa protokol yang diusulkan dapat berjalan.

  \item Simulasi dan Evaluasi\\
  Pada tahap ini, prototype yang telah dibuat dilakukan simulasi. Simulasi dilakukan dengan cara mengirimkan pesan berupa teks dan berkas biner. Hal ini dilakukan dengan menggunakan beberapa berkas dan teks yang berbeda. Setiap gambar yang telah diterima akan diperiksa integritasnya. Pesan terenkripsi juga akan dilakukan secara statistik dan visual. Pengujian keteracakan dilakukan dengan menggunakan NIST SP 800-22 \emph{randomness test}. Selain itu, akan dilakukan pengujian analisis entropi informasi untuk melihat keterkaitan antara satu blok dengan blok lainnya. Pengujian visual dilakukan dengan cara menampilkan gambar yang terenkripsi.
\end{enumerate}

\section{Sistematika Pembahasan}
Pembahasan pada buku laporan ini akan dibagi menjadi beberapa bab. Berikut ini merupakan sistematika penulisan pada laporan ini:

\begin{enumerate}
  \item Bab I Pendahuluan\\
  Pada bab ini, pembahasan akan berfokus pada pembahasan terkait motivasi penelitian yang tertuang pada latar belakang. Pada bab ini juga dijelaskan mengenai rumusan masalah serta tujuan dari penelitian yang dilakukan.

  \item Bab II Studi Literatur\\
  Bab ini berisi terkait teori-teori yang terkait dengan penelitian ini. Hal-hal yang dibahas pada bagian ini mencakup terkait sistem chaos, kriptografi kunci simetris AES, protokol-protokol komunikasi yang telah ada pada saat penelitian ini dilakukan, serta penelitian yang telah ada khususnya terkait dengan sistem chaos. Pada bab ini juga akan dibahas terkait protokol TLS standard.

  \item Bab III Analisis Masalah dan Rancangan Solusi\\
  Bab ini berfokus kepada analisis terkait ancaman-ancaman yang ada pada sebuah protokol komunikasi TLS. Pada bab ini juga dijelaskan terkait komposisi \emph{cipher suite} yang akan digunakan pada protokol TLS. Pada bab ini juga dijelaskan terkait rancangan protokol yang akan dibangun didasari oleh kebutuhan yang telah didefinisikan.

  \item Bab IV Implementasi Solusi dan Pengujian\\
  Pada bab ini, pembahasan akan berfokus pada implementasi dan pengujian dari protokol yang telah dijelaskan sebelumnya pada bab III. Pada bab ini akan dibahas terkait eksperimen terkait sistem \emph{chaos} yang telah dipilih. Pada bab ini juga akan dibahas terkait lingkungan protokol komunikasi yang diusulkan. Selain itu, pada bab ini akan dibahas terkait hasil pengujian protokol yang telah dibangun.

  \item Bab V Kesimpulan dan Saran\\
  Bab ini berisi terkait kesimpulan dari penelitian ini. Pada bab ini juga akan diberikan saran untuk penelitian yang selanjutnya.
\end{enumerate}
