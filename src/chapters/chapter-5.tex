\chapter{Kesimpulan dan Saran}

\section{Kesimpulan}
Berdasarkan hasil analisis, implementasi, dan pengujian yang telah dilakukan, diperoleh kesimpulan sebagai berikut:
\begin{enumerate}
  \item AES dengan kunci blok dinamis dapat diimplementasikan dengan memanfaatkan sistem acak CSPRNG berbasis sistem \emph{chaos} sebagai generator kunci blok. Cipher ini melakukan enkripsi dengan membagi pesan menjadi beberapa blok, selanjutnya blok tersebut dienkripsi dengan kunci blok yang dihasilkan dari \emph{chaos} dan counter yang dihasilkan dari \emph{chaos}. Proses dekripsi dilakukan dengan cara membentuk kunci dan counter menggunakan \emph{chaos}. Salah satu mode blok yang terbukti dapat digunakan untuk melakukan enkripsi ini adalah mode counter.
  \item Pengimplementasian Algoritma AES kunci blok dinamis pada berbasis sistem \emph{chaos} dapat dilakukan pada protokol TLSv1.2. Proses implementasi dilakukan dengan membuat sebuah \emph{cipher suite} pada protokol TLSv1.2. Sebelum melakukan proses enkripsi dan dekripsi, sistem harus menyimpan nilai \emph{state} awal dari sistem \emph{chaos}. Hal ini ditujukan agar bila terjadi kegagalan pada proses enkripsi dan dekripsi, sistem dapat mengembalikan nilai \emph{state} sistem \emph{chaos} ke posisi yang semula.
  \item Sistem CSPRNG berbasis \emph{chaos} dengan \emph{Sine-Henon map} memiliki keunggulan lebih dibandingkan sistem CSPRNG berbasis \emph{chaos} dengan \emph{Henon map}. keunggulan tersebut terletak pada adanya \emph{forward unpredictability} yang ditawarkan oleh sistem \emph{chaos} \emph{Sine-Henon map}. Hal ini dikarenakan sistem \emph{chaos} \emph{Sine-Henon map} memiliki parameter yang tidak bergantung pada hasil keluaran sistem CSPRNG sebelumnya. Hal ini membuat CSPRNG akan lebih sulit untuk ditebak.
  \item Hasil pengujian menunjukan bahwa pembangkit bilangan acak yang diusulkan memiliki kualitas yang setara dengan CSPRNG \texttt{urandom} pada sistem Linux. Hal ini dapat dilihat dari hasil uji statistik yang dilakukan dengan \emph{NIST Statistical Test} yang menunjukan bahwa pembangkit bilangan acak ini memiliki kualitas acak yang setara dengan CSPRNG \texttt{urandom} pada sistem Linux.
  \item Hasil pengujian menunjukan bahwa \emph{cipher} AES dengan kunci blok dinamis menunjukan bahwa ketahanannya setara dengan AES dengan kunci statis pada mode blok counter. Hal ini dapat dilihat dari hasil uji statistik yang dilakukan dengan \emph{NIST Statistical Test} yang menunjukan bahwa \emph{cipher} ini memiliki kualitas acak yang sama dengan \emph{cipher} AES dengan kunci statis. Selain itu, uji ketahanan \emph{cipher} menggunakan CCA dan MAD menunjukan bahwa \emph{cipher} ini memiliki ketahanan yang setara terhadap serangan analisis \emph{ciphertext} bila dibandingkan dengan \emph{cipher} AES dengan kunci statis.
  \item Berdasarkan hasil pengujian, protokol dapat berjalan dengan baik memanfaatkan \emph{cipher} kunci blok dinamis. Protokol ini juga mampu bertahan khususnya terhadap serangan \emph{replay}, serangan MITM, dan serangan \emph{tampering}.
\end{enumerate}

\section{Saran}
Adapun saran terkait pelaksanaan Tugas Akhir ini adalah sebagai berikut:
\begin{enumerate}
  \item Penerapan proses paralelisasi saat melakukan proses enkripsi dan dekripsi. Hal ini dilakukan untuk mempercepat proses pembacaan serta penulisan pesan.
  \item Implementasi \emph{cipher} AES kunci dinamis pada protokol TLSv1.3. Hal ini dapat dicapai dengan mengimplementasikan \emph{cipher} AES kunci dinamis ini untuk mendukung AEAD (\emph{Authenticated Encryption with Associated Data}).
\end{enumerate}
