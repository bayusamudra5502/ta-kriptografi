\chapter{Kesimpulan dan Saran}

\section{Kesimpulan}
Berdasarkan hasil analisis, implementasi, dan pengujian yang telah dilakukan, diperoleh kesimpulan sebagai berikut:
\begin{enumerate}
  \item Proses enkripsi pesan memanfaatkan Algoritma AES kunci blok dinamis yang aman dapat dilakukan dengan melibatkan CSPRNG berbasis \emph{chaos} pada pembangkitan kunci blok. Salah satu sistem \emph{chaos} yang cukup baik adalah Sine-Henon Map. Mode blok yang dapat digunakan pada \emph{cipher} adalah CTR. Enkripsi dilakukan dengan membagi pesan menjadi beberapa blok, selanjutnya blok tersebut dienkripsi dengan kunci blok yang dihasilkan dari \emph{chaos} dan counter yang dihasilkan dari \emph{chaos}. Proses dekripsi dilakukan dengan cara membentuk kunci dan counter menggunakan \emph{chaos}. 
  \item Pengimplementasian Algoritma AES kunci blok dinamis pada berbasis sistem \emph{chaos} dapat dilakukan pada protokol TLSv1.2. Proses implementasi dilakukan dengan membuat sebuah \emph{cipher suite} pada protokol TLSv1.2. Sebelum melakukan proses enkripsi dan dekripsi, sistem harus menyimpan nilai \emph{state} awal dari sistem \emph{chaos}. Hal ini ditujukan agar bila terjadi kegagalan pada proses enkripsi dan dekripsi, sistem dapat mengembalikan nilai \emph{state} sistem \emph{chaos} ke posisi yang semula.
  \item Hasil pengujian menunjukan bahwa pembangkit bilangan acak yang diusulkan memiliki kualitas yang setara dengan CSPRNG \texttt{urandom} pada sistem Linux. Selain itu, \emph{cipher} AES dengan kunci blok dinamis menunjukan bahwa kualitasnya setara dengan AES dengan kunci statis pada mode blok counter.
  \item Berdasarkan hasil pengujian, protokol dapat berjalan dengan baik memanfaatkan \emph{cipher} kunci blok dinamis. Protokol ini juga mampu bertahan khususnya terhadap serangan \emph{replay}, serangan MITM, dan serangan \emph{tampering}. 
\end{enumerate}

\section{Saran}
Adapun saran terkait pelaksanaan Tugas Akhir ini adalah sebagai berikut:
\begin{enumerate}
  \item Penerapan proses paralelisasi saat melakukan proses enkripsi dan dekripsi. Hal ini dilakukan untuk mempercepat proses pembacaan serta penulisan pesan.
  \item Implementasi \emph{cipher} AES kunci dinamis pada protokol TLSv1.3. Hal ini dapat dicapai dengan mengimplementasikan \emph{cipher} AES kunci dinamis ini untuk mendukung AEAD (\emph{Authenticated Encryption with Associated Data}). Hal ini dilakukan agar \emph{cipher} ini kompatibel dengan protokol TLSv1.3.
\end{enumerate}
