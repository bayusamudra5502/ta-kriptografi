\chapter{Studi Literatur}

\section{Pengantar Kriptografi}
Menurut \textcite{schneier1996}, Kriptografi merupakan ilmu pengetahuan dan seni yang berujuan untuk menjaga sebuah pesan tetap aman. Menurut \textcite{anderson2008}, Kriptografi dianggap sebagai pintu para pengembang keamanan untuk bertemu dengan ilmu matematika. Hal ini dapat terlihat bahwa banyak sekali algoritma kriptografi yang terkait dengan konsep matematika. Kriptografi dianggap juga sebagai seni. Menurut \textcite{munir2019}, hal ini dikarenakan dari pandangan sejarah berkembangnya kriptografi. Kriptografi ini terbentuk dikarenakan adanya keinginan untuk merahasiakan sebuah pesan. Tentu saja, setiap orang memiliki ciri khas serta caranya tersendiri untuk menyandikan sebuah pesan. Oleh karena itu, kriptografi dapat dianggap sebagai seni untuk merahasiakan sebuah pesan.

\subsection{Fungsi Kriptografi}
Kriptografi 

\subsection{Pesan dan cipherteks}
Dalam dunia kriptografi, Pesan merupakan sebuah data yang dapat dibaca dan dimengerti. Pesan ini 

\section{Pengantar Jaringan Komputer}
Testing 

\subsection{TCP Protocol}
Testing

\section{\emph{Advanced Encryption Standard (AES)}}
Testing

\section{Teori Chaos}
Testing

\section{Pembangkit Bilangan Acak yang Aman untuk Kriptografi}
Testing

\subsection{CSRPNG berbasis Chaos}
Testing

\section{Fungsi Hash Satu Arah}
Testing

\subsection{\emph{Secure Hash Algorithm (SHA)}}
Testing

\section{Penelitian Terkait}
Testing

\subsection{Sinkronisasi Kunci Dinamis berbasis Chaos den Aplikasinya pada Pengenkripsian Gambar dengan Algoritma AES yang diimprovisasi}
Testing

\subsection{Enkripsi Gambar dan Analisisnya memanfaatkan Algoritma AES dinamis}
Testing

\subsection{Peningkatan Keamanan Enkripsi AES menggunakan Algoritma \emph{Salt} Dinamis}
Testing
