% ---------------------------------------------------%
% DISCLAIMER: ADA KEMUNGKINAN BAB 3 STRUKTUR YG DIBAHAS BERBEDA, JADI DI BAWAH ADA CONTOH
% BAB 3 KATING. MODIF SAJA SESUAI KEBUTUHAN

% ---------------------------------------------------%

% \chapter{Analisis dan Perancangan}
\chapter{Analisis dan Rencana Penelitian}

\section{Analisis Masalah}

Dalam mengimplementasikan protokol komunikasi dengan memanfaatkan cipher dengan kunci dinamis berbasis \emph{chaos}, terdapat beberapa permasalahan yang muncul. Permasalahan pertama adalah metode untuk melakukan sinkronisasi sistem \emph{chaos} dengan jumlah komunikasi yang konstan. Sistem \emph{chaos} ini digunakan sebagai \emph{state} yang dimanfaatkan untuk mengatur keteracakan pada hasil enkripsi menggunakan AES, sehingga diperlukan sistem \emph{chaos} yang tersinkronisasi. Selain itu, komunikasi memungkinkan dilakukan pada jaringan yang kurang reliabel.

Permasalahan kedua yang muncul dengan mengimplementasikan protokol ini adalah metode untuk melakukan enkripsi tanpa perlu mengubah internal dari AES standar. Saat ini, proses enkripsi memanfaatkan AES dapat dijalankan dalam level instruksi mesin seperti yang tertulis pada \textcite{gueron2010}. Pengubahan internal AES standar dapat menyebabkan instruksi AES standar tidak bisa digunakan lagi.

Permasalahan ketiga yang muncul adalah metode untuk mengirimkan pesan semetode \emph{full-duplex}. Saat ini, semetode lumrah komunikasi antar dua buah pihak dapat dilakukan semetode \emph{full-duplex}. Akan tetapi, pengiriman memanfaatkan kunci yang dinamis memerlukan kedua belah pihak menyimpan state yang sama untuk melakukan enkripsi dan dekripsi. Oleh karena itu, diperlukan sebuah metode agar proses ini dapat dilakukan semetode sinkron tanpa menghilangkan sifat \emph{full-duplex}.

Dalam komunikasi pada jaringan yang tidak aman, dimungkinkan terjadinya penerimaan data dari pihak yang tidak sah. Hal ini dapat menimbulkan permasalahan pada protokol dikarenakan dapat mengancam ketersinambungan \emph{state}. Oleh karena itu, diperlukan metode untuk memastikan pesan terotentikasi dan metode penanggulangan apabila proses otentikasi gagal.

\section{Analisis Solusi} 

Berdasarkan \textcite{lin2021}, proses sinkronisasi dapat dilakukan dengan metode mengirimkan parameter sinkronisasi untuk sistem chaos. Kekurangan metode ini adalah jumlah komunikasi yang diperlukan cukup banyak sehingga kurang efektif bila dilakukan pada jaringan kurang reliabel. Metode lainnya yang dapat dilakukan adalah dengan memanfaatkan algoritma pertukaran kunci Diffie-Hellman sebagaimana yang dilakukan pada protokol SSL sebagaimana yang dijelaskan pada \textcite{munir2019}.

Proses pengenkripsian pesan secara dinamis dapat dilakukan dengan memanfaatkan kunci blok. Dengan melakukan hal ini, proses enkripsi pada internal AES tidak perlu berubah. Selain itu, keteracakan kunci blok akan menjamin hasil enkripsi akan berbeda walaupun memiliki pesan yang sama sebagaimana pengujian yang dilakukan oleh \textcite{lin2021}.

Untuk mengirimkan data secara \emph{full-duplex}, dapat dilakukan dengan metode memisahkan sistem \emph{chaos} untuk melakukan pengiriman dan penerimaan data. Hal ini dapat menjamin pengiriman data dapat dilakukan secara bersamaan. Bagi pengirim, Nilai sistem \emph{chaos} akan diupdate jika proses pengiriman telah mendapatkan konfirmasi. Bagi penerima, Nilai sistem \emph{chaos} akan diupdate jika proses pengiriman telah mendapatkan blok pesan baru dari pengirim.

Dalam melakukan verifikasi pesan, dapat dilakukan sebuah metode yaitu dengan cara memanfaatkan HMAC. Setiap pesan yang diterima perlu diperiksa melalui HMAC dengan kunci blok yang digunakan pada pendeskripsian blok sebelumnya. Apabila pesan yang diterima sah, sistem \emph{chaos} akan diperbaharui. Nilai sistem \emph{chaos} selanjutnya akan digunakan untuk mendekripsi pesan tersebut.

\section{Gambaran Solusi}

Proses enkripsi dan dekripsi yang dilakukan pada protokol yang akan dibangun digambarkan pada gambar \ref{fig:solution.encrypt} dan \ref{fig:solution.decrypt}. Mode blok enkripsi yang akan digunakan adalah \emph{cipher block chaining} (CBC). Setiap blok pesan yang dikirimkan akan diberikan sebuah MAC sebagai otentikasi dari pesan yang dikirimkan. Apabila dalam pengiriman blok mengalami kegagalan dikarenakan kegagalan otentikasi, pesan akan diretransmisi ulang. Pembangkitan kunci pada protokol ini akan memanfaatkan sistem \emph{chaos} yang didefinisikan pada persamaan \ref{eq:lin.henon.master}.

\begin{figure}[!h]
  \centering
  \includegraphics[width=250px]{chapters/res/chapter-3/img/encrypt.png}
  \caption{Enkripsi pesan secara dinamis untuk pesan yang dikirimkan dari Alice menuju Bob} \label{fig:solution.encrypt}
\end{figure}

\begin{figure}[!h]
  \centering
  \includegraphics[width=250px]{chapters/res/chapter-3/img/decrypt.png}
  \caption{Dekripsi dan validasi pesan secara dinamis untuk pesan yang dikirimkan dari Alice menuju Bob} \label{fig:solution.decrypt}
\end{figure}

Pada proses awal komunikasi, akan dibentuk dua buah sistem \emph{chaos} melalui mekanisme pertukaran kunci \emph{diffie-hellman}. Kedua belah pihak akan melakukan pertukaran kunci memanfaatkan \emph{Diffie-Hellman}. Nilai hasil pertukaran kunci akan digunakan sebagai parameter awal nilai $(X_1, Y_1, Z_1)$ untuk sistem $S_1$ dan $(A_1, B_1, C_1)$ pada sistem chaos $S_2$. Hal ini dapat diperoleh dengan melakukan \emph{sclicing} bit hasil pertukaran kunci. Parameter $(X_1, Y_1, Z_1)$ dan $(A_1, B_1, C_1)$ bersifat rahasia. Penentuan nilai $IV$ akan ditentukan oleh pengirim pesan. Nilai ini akan dibuat secara acak memanfaatkan \emph{secure random} yang disediakan pada sistem operasi. Nilai $IV$  tidak bersifat rahasia.

Saat proses pertukaran kunci berhasil dan sistem \emph{chaos} berhasil dibuat, proses komunikasi akan dimulai. Misalkan, terdapat dua pihak yang melakukan komunikasi, yaitu Alice dan Bob. Sistem \emph{chaos} $S_1$ akan digunakan saat pengiriman data berlangsung dari Alice menuju Bob.  Sebaliknya, sistem \emph{chaos} $S_2$ akan digunakan saat pengiriman data berlangsung dari Bob menuju Alice.

Misalkan, pengriman data berlangsung dari Alice menuju Bob. Alice akan mengenkripsi blok $M_i$ menggunakan nilai $K_i$. Nilai $K_i$ dihasilkan berdasarkan hasil konversi nilai $X_i$ pada sistem $S_1$ menjadi bilangan bulat melalui fungsi $f$. Setelah pesan dienkripsi menjadi $C_i$, akan dihitung nilai MAC dengan kunci $K_i$ untuk $C_i$ dan disematkan pada $C_i$. Proses enkripsi berlangsung sebagaimana gambar \ref{fig:solution.encrypt}. Nilai $(X_{i+1}, Y_{i+1}, Z_{i+1})$ pada Alice akan dihitung apabila pesan $M_i$ berhasil dienkripsi menjadi $C_i$.  

Saat bob menerima pesan, bob akan memeriksa keabsahan data melalui MAC. Bob akan menghitung nilai $(X_i, Y_i, Z_i)$ dan membantuk nilai $K_i$. Setelah itu, Bob akan menghitung MAC $C_i$ dan membandingkannya dengan nilai MAC yang dikirimkan oleh Alice. Apabila nilai MAC sama, nilai $(X_{i-1}, Y_{i-1}, Z_{i-1})$ akan dihapus pada Bob. Selanjutnya, proses dekripsi dilakukan sebagaimana yang dijelaskan pada gambar \ref{fig:solution.decrypt}.
